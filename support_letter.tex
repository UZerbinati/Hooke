\documentclass[11pt,a4paper]{article}

\usepackage[margin=2.5cm]{geometry}
\usepackage{parskip}
\usepackage{times}
\usepackage{hyperref}

\setlength{\parindent}{0pt}

\begin{document}

\begin{center}
    {\Large \textbf{Supporting Letter -- Umberto Zerbinati}}\\[4pt]
    Vacancy Reference: 182821\\
    Primary Research Group: Numerical Analysis\\
    Secondary Research Group: OCIAM
\end{center}

\vspace{1em}

Dear Members of the Selection Committee,

I write to apply for the Hooke Research Fellowship in the Mathematical Institute.
My research lies at the interface of \textbf{partial differential equations, numerical analysis, scientific computing and thermodynamics},
with particular emphasis on multiscale models, kinetic descriptions of ordered fluids, and structure-preserving numerical methods.
I identify \textbf{Numerical Analysis} as the research group closest to my work, with \textbf{OCIAM} as a natural secondary group given the applied, model-driven nature of my research programme.

I have successfully passed my Confirmation of Status, and I expect to submit my DPhil thesis by the end of June 2025.


\section*{Research excellence, independence, and achievements}

My research to date demonstrates independence, originality, and a consistently strong output.
I have five published papers in \textit{Multiscale Modeling \& Simulation, Computer Methods in Applied Mechanics and Engineering, Physical Review E, Calcolo}, and \textit{Journal of Open Source Software}.
I also have four submitted papers to \textit{Multiscale Modeling \& Simulation, the IMA Journal of Numerical Analysis, Computers \& Mathematics with Applications}, and the \textit{SIAM Journal on Scientific Computing}.
My contributions span the derivation and analysis of models for complex fluids arising in many scientific applications, such as liquid crystals.

Boltzmann first derived the kinetic theory for monatomic gases, and eighty years later Curtiss extended it to polyatomic molecules.
I addressed the question of developing kinetic theories for fluids possessing an arbitrary internal microstructure and established a general framework based on Capriz's order-parameter manifold.

I also investigated the derivation of hydrodynamic equations governing complex fluids at microscopic scales, starting from this generalised kinetic framework.
In the particular case of rod-like molecules, our framework yields the Boltzmann–Curtiss equation, and I have developed a new moment-closure technique—drawing inspiration from rational thermodynamics—to recover the compressible Leslie--Ericksen equations.
Quoting from the MathSciNet review (\url{https://mathscinet.ams.org/mathscinet/article?mr=4834435}) of our work by Z. Guo:
“[The paper] establishes a principled kinetic foundation for compressible nematic gas dynamics and paves the way for deriving the long-sought viscous, compressible Leslie--Ericksen equations in future research.”

I also recently derived the nematic Helmholtz--Korteweg equations, which describe acoustic propagation in liquid crystals, predict anisotropic sound speed, and reveal the existence of a new class of tunable acoustic resonators.
I have also designed provably convergent schemes for this equation, which combines the challenges of a Helmholtz-type operator with those of a biharmonic problem.

As a side project, I developed a novel numerical method, the lightning virtual element method, which overcomes several fundamental drawbacks of the classical virtual element method.
In parallel, I have studied preconditioned normal equations in the context of preconditioning large-scale PDE solvers.

Complementing my theoretical and computational research, I actively develop scientific software.
I contribute to Firedrake and NGSolve, and I maintain my own open-source library, ngsPETSC.

This integration of modelling, rigorous analysis, and high-performance computing is a central theme of my research.
Across these collaborations I have played central roles in mathematical formulation, analytical development, and software implementation.

\section*{Academic engagement: conferences, seminars, community}

I am an active participant in the international applied mathematics community.
I've been invited to give departmental colloquiums at the University of Prague by Josef Male, the University of Oslo by Thomas Surowiec and King Abdullah University of Science and Technology by Daniele Boffi.
I have been invited independently of my advisor to CIRM, the Newton Institute and the University of Vienna to collaborate with Paul Stocker and to a two research conferences organised by Kaibo Hu at  Tsinghua University and the Chinese Academy of Sciences.

In the past year alone, I have attended {21 conferences}, presenting my work across Europe and North America.
I have also taken part in workshops at institutions including {CIRM} and given seminars at multiple applied mathematics groups across both continents.

A complete list of presentations and slides is available at \texttt{uzerbinati.eu/slides}.

In addition, I am the {founder and co-organiser of the first three editions of the Young Applied Mathematicians Conference (YAMC) in Italy}, which has grown into a significant platform for early-career researchers in applied and computational mathematics.

I am also an external collaborator with Lawrence Livermore National Laboratory on topics related to numerical methods for the kinetic theory of dense gases.

\section*{Teaching and supervision}

I have substantial teaching experience across colleges and the Mathematical Institute, including tutorials and classes in
\textit{Numerical Analysis (A7)}, \textit{Applied PDEs (B5.2)}, \textit{Metric Spaces and Complex Analysis (A2)},
and \textit{Numerical Linear Algebra (C6.1)}.

In 2025, I was appointed {Stipendiary Lecturer in Applied Mathematics at Oriel College}.
I also delivered a {four-part guest lecture series} at the University of Edinburgh on spectral theory and spectral practice,
invited by Prof.\ Kaibo Hu.
These experiences demonstrate my ability to teach effectively in small groups and contribute both to Oxford’s tutorial system and to the department’s teaching activities.

\section*{Collegiality, inclusivity, and service}

I am committed to fostering inclusive, collaborative research environments.
My academic experience across Italy, Austria, Saudi Arabia, and the UK has deepened my appreciation for respectful, supportive interactions.
My involvement with YAMC reflects my dedication to community-building and supporting junior researchers.
I aim to contribute to the Mathematical Institute through mentoring, seminar participation, and software-driven collaborations.

\section*{Conclusion}

I believe that my research record, publication output, teaching experience, and future research plans align strongly with the aims of the Hooke Research Fellowship.
The opportunity to pursue my research vision within the Mathematical Institute would be invaluable.

\vspace{1em}

Yours sincerely,\\
{Umberto Zerbinati}

\end{document}

