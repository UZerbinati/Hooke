\documentclass[11pt,a4paper]{article}

\usepackage[margin=2.5cm]{geometry}
\usepackage{parskip}
\usepackage{times}
\usepackage{hyperref}

\setlength{\parindent}{0pt}

\begin{document}

\begin{center}
    {\Large \textbf{Supporting Letter -- Umberto Zerbinati}}\\[4pt]
    Vacancy Reference: 182821\\
    Primary Research Group: Numerical Analysis\\
    Secondary Research Group: OCIAM
\end{center}

\vspace{1em}

Dear Members of the Selection Committee,

I write to apply for the Hooke Research Fellowship in the Mathematical Institute.
My research lies at the interface of \textbf{partial differential equations, numerical analysis, scientific computing and thermodynamics},
with particular emphasis on multiscale models, kinetic descriptions of ordered fluids, and structure-preserving numerical methods.
I identify \textbf{Numerical Analysis} as the research group closest to my work, with \textbf{OCIAM} as a natural secondary group given the applied, model-driven nature of my research programme.

I have successfully passed my Confirmation of Status, and I expect to submit my DPhil thesis by the end of June 2025.


\section*{Research excellence, independence, and achievements}

My research to date demonstrates independence, originality, and a consistently strong output.
I have five published papers in \textit{Multiscale Modeling \& Simulation, Computer Methods in Applied Mechanics and Engineering, Physical Review E, Calcolo}, and \textit{Journal of Open Source Software}.
I also have four submitted papers to \textit{Multiscale Modeling \& Simulation, the IMA Journal of Numerical Analysis, Computers \& Mathematics with Applications}, and the \textit{SIAM Journal on Scientific Computing}.
My contributions span the derivation and analysis of models for complex fluids arising in many scientific applications, such as liquid crystals.

Boltzmann first derived the kinetic theory for monatomic gases in 1872, and eighty years later Curtiss extended it to polyatomic molecules.
A main goal of my thesis is to extend their work to develop kinetic theories for fluids with arbitrary internal microstructure. The key mathematical tool in this is the order parameter manifold introduced by Capriz.

I also investigated the derivation of hydrodynamic equations governing complex fluids at microscopic scales, starting from this generalised kinetic framework.
In the particular case of rod-like molecules, our framework yields the Boltzmann–Curtiss equation, and I have developed a new moment-closure technique—drawing inspiration from rational thermodynamics—to derive a compressible variant of the Leslie--Ericksen equations.
%Quoting from the MathSciNet review (\url{https://mathscinet.ams.org/mathscinet/article?mr=4834435}) of our work by Z. Guo:
%“[The paper] establishes a principled kinetic foundation for compressible nematic gas dynamics and paves the way for deriving the long-sought viscous, compressible Leslie--Ericksen equations in future research.”

I also recently derived the nematic Helmholtz--Korteweg equations, which describe acoustic propagation in liquid crystals, predict anisotropic sound speeds, and predict the existence of a new class of tunable acoustic resonators.
I have also designed provably convergent schemes for this equation, which combines the challenges of a Helmholtz-type operator with those of a biharmonic problem.

As a side project, I developed a novel numerical method, the lightning virtual element method, which overcomes several fundamental drawbacks of the now-classical virtual element method.
In parallel, I have studied preconditioned normal equations in the context of preconditioning large-scale PDE solvers.

Complementing my theoretical and computational research, I actively develop scientific software.
I contribute to Firedrake and NGSolve, and I maintain my own open-source library, ngsPETSC. This is now in widespread use in the Firedrake and NGSolve communities.

This integration of modelling, rigorous analysis, and high-performance computing is a central theme of my research.
Across these collaborations I have played central roles in mathematical formulation, analytical development, and software implementation.

\section*{Academic engagement: conferences, seminars, community}

I am an active, independent participant in the international applied mathematics community.
I have been invited to give departmental seminars at the University of Prague by Josef M\'alek, the University of Oslo by Thomas Surowiec, the University of Bath by Tristen Pryer, the University of Catania by Giovanni Russo, and King Abdullah University of Science and Technology (KAUST) by Daniele Boffi.

I have been invited (independent of my advisor) to CIRM twice and to the Newton Institute.

I have received six invitations to participate in minisymposia at international conferences. Again, these invitations were independent of my advisor.
In addition, I have contributed talks at tens of other conferences.
A complete list of presentations and slides is available at \texttt{uzerbinati.eu/slides}.


%and to two research conferences organised by Kaibo Hu at  Tsinghua University and the Chinese Academy of Sciences.

%In the past year alone, I have attended {21 conferences}, presenting my work across Europe and North America.
%I have also taken part in workshops at institutions including {CIRM} and given seminars at multiple applied mathematics groups across both continents.

In addition, I am the {founder and main organiser of the first three editions of the Young Applied Mathematicians Conference (YAMC) in Italy}, which has grown into a significant platform for early-career Italian researchers in applied and computational mathematics.

I also hold an unpaid appointment at Lawrence Livermore National Laboratory on numerical methods for the kinetic theory of dense gases. This gives me access to their supercomputing resources.

\section*{Teaching and supervision}

I was invited to give a four-part special lecture series on spectral theory and spectral practice by Kaibo Hu while he was at the University of Edinburgh. All other lecturers so invited (Yuwen Li, Snorre Christiansen, Evan Gawlik) are permanent faculty. In the Mathematical Institute I gave two lectures for \emph{Numerical Analysis (A7)}, standing in for Charles Parker while he attended a conference.

In 2025, I was appointed {Stipendiary Lecturer in Applied Mathematics at Oriel College}. I teach all of Patrick Farrell's college tutorials while he serves as a visiting professor at Charles University. I have given tutorials in \emph{Dynamics}, \emph{Differential Equations I}, \emph{Mathematical Biology}, \emph{Calculus of Variations}, \emph{Introductory Calculus}, \emph{Multivariable Calculus}, \emph{Metric Spaces \& Complex Analysis}, and \emph{Numerical Analysis}. I have given departmental classes in \textit{Applied PDEs (B5.2)} and
\textit{Numerical Linear Algebra (C6.1)}.

These experiences demonstrate my ability and passion for teaching, in both small groups and in lectures.

\section*{Collegiality, inclusivity, and service}

I am committed to fostering inclusive, collaborative research environments.
My academic experience across Italy, Austria, Saudi Arabia, and the UK has deepened my appreciation for respectful, supportive interactions.
My involvement with YAMC reflects my dedication to community-building and supporting junior researchers.
I aim to contribute to the Mathematical Institute through mentoring, seminar participation, and software-driven collaborations.

\section*{Conclusion}

I believe that my research record, publication output, teaching experience, and future research plans align strongly with the selection criteria for the Hooke Research Fellowship.
While I have informal postdoctoral offers from Oslo, Vienna, and KAUST, my first choice is to remain in Oxford. The research environment here is unique: there is nowhere else in the world that combines the level of expertise in numerical analysis, kinetic theory, and partial differential equations. This is the best place in the world to pursue my research.

%\vspace{1em}

Yours sincerely,\\
{Umberto Zerbinati}

\end{document}

