\documentclass[11pt,a4paper]{article}

\usepackage[margin=2.5cm]{geometry}
\usepackage{parskip}
\usepackage{times}

\setlength{\parindent}{0pt}

\begin{document}

\begin{center}
    {\Large \textbf{Supporting Letter -- Umberto Zerbinati}}\\[4pt]
    Vacancy Reference: 182821\\
    Primary Research Group: Numerical Analysis\\
    Secondary Research Group: OCIAM
\end{center}

\vspace{1em}

Dear Members of the Selection Committee,

I am writing to apply for the Hooke Research Fellowship in the Mathematical Institute.
My research lies at the interface of \textbf{partial differential equations, numerical analysis, and scientific computing},
with particular emphasis on multiscale models, kinetic descriptions of ordered fluids, and structure-preserving numerical methods.
I identify \textbf{Numerical Analysis} as the research group closest to my work, with \textbf{OCIAM} as a natural secondary group given the applied, model-driven nature of my research programme.

I have successfully passed my Confirmation of Status at the University of Oxford, and I expect to submit my DPhil thesis by the end of June 2025.


\section*{Research excellence, independence, and achievements}

My research to date demonstrates independence, originality, and a consistently strong output.
I have produced {14 research outputs}, including journal papers, proceedings articles, and preprints,
with publications in venues such as \textit{Computer Methods in Applied Mechanics and Engineering}, \textit{Multiscale Modeling \& Simulation} and \textit{Physical Review E}.
My contributions span the derivation and analysis of models for complex fluids arising from kinetic theory and continuum mechanics,
as well as the development of robust numerical schemes for the resulting PDE systems.

In my recent work, I developed a kinetic-theory framework for ordered fluids and proposed and analysed the \emph{nematic Helmholtz--Korteweg equations}, which describe acoustic propagation in liquid crystals.
I have also designed efficient numerical schemes for these equations, contributed to the numerical analysis of unfitted finite element methods, and helped develop a novel numerical method known as the \emph{lightning virtual element method}.
In parallel, I have studied the \emph{preconditioned normal equations} in the context of preconditioning large-scale PDE solvers.
Across these collaborations I have played central roles in mathematical formulation, analytical development, and software implementation.

Complementing my research work, I also actively develop scientific software.
I contribute to \texttt{Firedrake} and \texttt{NGSolve}, and I maintain my own open-source library \texttt{ngsPETSC}.
This integration of modelling, rigorous analysis, and high-performance computing is a central theme of my research.

\section*{Academic engagement: conferences, seminars, community}

I am an active participant in the international applied mathematics community.
In the past year alone, I have attended {21 conferences}, presenting my work across Europe and North America.
I have also taken part in workshops at institutions including {CIRM} and given seminars at multiple applied mathematics groups across both continents.

A complete list of presentations and slides is available at \texttt{uzerbinati.eu/slides}.

In addition, I am the {founder and co-organiser of the first three editions of the Young Applied Mathematicians Conference (YAMC)},
which has grown into a significant platform for early-career researchers in applied and computational mathematics.

\section*{Teaching and supervision}

I have substantial teaching experience across colleges and the Mathematical Institute, including tutorials and classes in
\textit{Numerical Analysis (A7)}, \textit{Applied PDEs (B5.2)}, \textit{Metric Spaces and Complex Analysis (A2)},
and \textit{Numerical Linear Algebra (C6.1)}.

In 2025, I was appointed {Stipendiary Lecturer in Applied Mathematics at Oriel College}.
I also delivered a {four-part guest lecture series} at the University of Edinburgh on spectral theory and spectral practice,
invited by Prof.\ Kaibo Hu.
These experiences demonstrate my ability to teach effectively in small groups and contribute both to Oxford’s tutorial system and to the department’s teaching activities.

\section*{Collegiality, inclusivity, and service}

I am committed to fostering inclusive, collaborative research environments.
My academic experience across Italy, Austria, Saudi Arabia, and the UK has deepened my appreciation for respectful, supportive interactions.
My involvement with YAMC reflects my dedication to community-building and supporting junior researchers.
I aim to contribute to the Mathematical Institute through mentoring, seminar participation, and software-driven collaborations.

\section*{Conclusion}

I believe that my research record, publication output, teaching experience, and future research plans align strongly with the aims of the Hooke Research Fellowship.
The opportunity to pursue my research vision within the Mathematical Institute would be invaluable.

\vspace{1em}

Yours sincerely,\\
{Umberto Zerbinati}

\end{document}

