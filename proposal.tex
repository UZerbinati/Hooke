
\documentclass[11pt,a4paper]{article}
\usepackage{fullpage}
\usepackage{parskip}
\usepackage{times}

\setlength{\parindent}{0pt}

\begin{document}

\begin{center}
    {\Large \textbf{
        Algebraic Methods in Kinetic Theory: Analysis, Modelling and Numerics
    }}\\[4pt]
    Applicant: Umberto Zerbinati, \qquad Vacancy Reference: 182821\\
\end{center}

Many complex fluids such as liquid crystals, ferrofluids, and polymer solutions possess an internal \emph{microstructure} that endows them with orientation or ordering at the microscale.
A powerful framework to include such effects is \textbf{Capriz's order parameter manifold} concept \cite{capriz}, which describes using the notion of a manifold and an associated Lie group action, the set of internal states of the fluid, such as orientation, polarisation and conformation.
Historically, Capriz's idea has been used to formulate phenomenological continuum theories of ordered fluids.
In my research, I will extend this concept to a kinetic theory setting and leverage algebraic and geometric properties of the order parameter manifold to derive new models, analytical insights, and numerical methods for fluids with microstructure. 
The project is organized into three interconnected strands, described below.

\textbf{1. Kinetic Theory}\\
In a recent work (with J.~A.~Carrillo, P.~E.~Farrell, and A.~Medaglia), we introduced a kinetic equation for microstructured fluids (the \emph{CFMZ kinetic model}) that extends the classical Boltzmann picture by including the appropriate generalized angular momenta associated with the microstructure \cite{cfmz}.
Our theory yields a mesoscopic model for any continuum with microstructure characterized by a given order parameter manifold.

\textbf{Hydrodynamic Limits and Closure:} We will employ novel moment-closure techniques to derive macroscopic equations (analogous to Navier--Stokes or the Leslie--Ericksen equations) that faithfully capture microstructural effects. In collaboration with G.~Russo and P.~E.~Farrell, I have already devised a novel closure strategy for a simplified Boltzmann--Curtiss system, inspired by Rational Thermodynamics, which successfully reproduces the anisotropic stress of nematic liquid crystals and yields a compressible variant of the Leslie--Ericksen equations. I plan to extend this approach, working with J.~Málek, to obtain thermodynamically consistent continuum models for arbitrary microstructures.
    
\textbf{Algebraic and Geometric Properties of the Manifold} We plan to examine the impact of the manifold's geometry (curvature, topology) on the kinetic theory. In particular, I will investigate whether certain geometric conditions (e.g. convexity or curvature bounds on the manifold) allow a separation of scales on the manifold, introducing slowly-varying and rapidly-varying coordinates in the BBGKY derviation presented in \cite{cfmz}.
The aim is to exploit this separation to simplify the kinetic equation, possibly eliminating any mean-field Vlasov term in favor of modified collision operators. 
Furthermore, in \cite{cfmz} we showed that if this action is transitive, then the only collision invariants of the kinetic equation are the classical microscopic invariants (mass, linear momentum, generalised angular momentum and energy).
I will explore further: for example, using the orbit stabiliser theorem the consequence that the symmetries of the Lie group has on the kinetic theory of fluids with internal microstructures.
    
\textbf{Diffuse-Interface Limit:} Following ideas from Onsager’s theory of liquid crystals and the work of V.~Giovangigli on diffuse-interface models, I will explore the derivation of diffuse-interface models from a kinetic description, also in the context of fluids with microstructures.
I plan to derive a diffuse-interface continuum model directly from the CFMZ kinetic equation following the ideas from the kinetic theory of dense gases.
This model would include stress contributions analogous to the Korteweg stress and coupling terms between the density variations and microstructural quantities.

\textbf{2. Liquid Crystals}
Liquid crystals (LCs) serve as an ideal testbed for the order-parameter-manifold approach and offer exciting application opportunities. Nematic liquid crystals consist of rod-like molecules that exhibit orientational order (a broken rotational symmetry), and their continuum behavior cannot be captured by standard fluids models. This part of the project will apply our theoretical developments to nematic LCs, particularly in the context of \textbf{nemato-acoustics} (sound propagation in liquid crystalline media). In collaboration with P.~E.~Farrell, we have developed a new continuum model for acoustic waves in LCs \cite{helmholtzKorteweg}.

\textbf{Nematic Helmholtz--Korteweg Equations:} Building on Virga’s nematoacoustics theory, that regards nematic LCs as Korteweg fluids with an extra coupling between density gradients and the nematic director field \cite{virga}, we derived the \emph{nematic Helmholtz--Korteweg equations}.
These govern time-harmonic acoustic wave propagation in a compressible nematic fluid, incorporating both elastic nematic effects and capillarity. Together with P.~Farrell and T.~van Beeck, we proved the well-posedness of these equations and developed a provably convergent finite element scheme for their numerical solution. Our simulations have reproduced experimental observations and even predicted new phenomena in liquid crystal acoustics, suggesting designs for tunable acoustic devices (e.g. variable-focus acoustic lenses or resonators controlled by the LC orientation).
We now intend to study wave propagation across interfaces and defects in liquid crystals. One problem of particular interest is the \emph{transmission problem} for the nematic Helmholtz--Korteweg equations: how an acoustic wave traverses a boundary between distinct liquid crystalline domains or between an LC and a standard fluid. Understanding transmission and reflection in these scenarios is crucial for real-life applications like display technologies or acoustic sensors using LCs.
    
\textbf{Derivation of Continuum Parameters from Kinetic Model:} A key question is how the phenomenological parameters in Virga's constitutive relations (such as the strength of coupling between density and director, or the interfacial tension coefficients) relate to underlying molecular properties. Using the diffuse-interface kinetic model (from the previous section), I will seek to derive or estimate these constitutive parameters from first principles. In particular, I will identify the scaling regimes under which the kinetic theory for rod-like molecules reduces to the nematic Korteweg stress form. This effort, likely involving asymptotic analysis and numerical fitting, would greatly enhance the predictive power of continuum LC models and be of immense value to industry, enabling material designers to tune acoustic properties by altering molecular features.
    
\textbf{Inverse Problems and Bio-imaging:} Another frontier is the inverse problem in nemato-acoustics: determining the internal alignment or material parameters of a liquid-crystalline medium from acoustic measurements. Since many biological tissues (muscle fibers, collagen matrices, etc.) exhibit liquid-crystal-like ordering, solving this inverse problem could open new modalities in non-invasive imaging (similar to ultrasound, but enhanced by anisotropic response). I plan to investigate whether the nematic Helmholtz--Korteweg model parameters can be reconstructed from, say, acoustic scattering data. This will involve developing appropriate optimization or machine-learning frameworks constrained by our PDE model. I have engaged industrial partners \textbf{ZOEEN S.R.L.} and \textbf{Quanta System S.p.A.}, who have expressed strong interest in exploring these effects for biomedical imaging technologies. Our goal is to pave the way for acoustics-based diagnostics that exploit the unique signatures of liquid-crystalline structures in tissue.

\textbf{3. Numerical Methods}
The complex PDE systems arising from the above research (both kinetic equations in high-dimensional phase space and continuum models with unconventional couplings) require advanced numerical methods. A significant part of this project is devoted to developing \textbf{structure \& asymptotic preserving algorithms} that respect the physical invariants (mass, momentum, energy, angular momentum) and efficiently handle the multiscale nature of the problems. My background in numerical analysis and scientific computing (including contributions to libraries like \texttt{Firedrake} and \texttt{NGSolve}) will be instrumental here. Key objectives include:

\textbf{Angular Momentum Preserving FE Schemes:} Design finite element methods that preserve angular momentum for liquid crystal hydrodynamics. In contrast to standard fluids (where ensuring a symmetric stress tensor weakly usually suffices to conserve angular momentum), we found (with C.~Parker and P.~Brubeck) that naive mixed finite element discretizations of nematic stress can lead to a loss of angular momentum conservation if the stress symmetry is only imposed weakly.
To address this, I will collaborate with J.~Schöberl and J.~Gopalakrishnan on developing a new class of mixed finite element or hybrid schemes that enforce the symmetry (or skew-symmetry) of certain stress components strongly.
These schemes will be tested on benchmark LC flow problems to verify that they correctly capture the transfer between linear and angular momentum.
Beyond classical angular momentum, the CFMZ kinetic theory introduces a \emph{generalized angular momentum} associated with the microstructure. I plan to investigate numerical methods that conserve this quantity.
    
\textbf{DSMC for Ordered Fluids:} Develop particle-based simulation tools for the kinetic equations. Direct Simulation Monte Carlo (DSMC) is a classical technique for the Boltzmann equation \cite{Pareschi}; however, applying DSMC to an extended kinetic model with orientation degrees of freedom is largely unexplored. Together with A.~Medaglia, I am currently implementing a DSMC algorithm for the CFMZ equation specializing to liquid crystals (where particles carry an orientation in addition to velocity). The project will generalize the DSMC methodology to other microstructured fluids (for example, particulate suspensions or polymer solutions) by incorporating different order parameter manifolds. These simulations will not only provide insight into the kinetics (especially in regimes difficult for analysis) but also supply data to inform moment closures and continuum model parameters.

\textbf{Deflation Techniques for Symmetric Solutions:} Many nonlinear problems in our field exhibit multiple solutions due to symmetry-breaking.
For instance the Oseen--Frank energy functional admits multiple local minimisers all corresponding to physically valid nematic configurations.
To systematically compute multiple solutions, I will employ and extend \emph{deflation} methods.
In particular, I plan to develop a deflation strategy that can \emph{quotient out} group orbits: once one solution is found, a modified solver will seek another solution that is not just a trivial rotation/translation of the first. This will involve incorporating the information from the Lie group action associated to the order parameter manifold into the deflation operator. Such techniques will be invaluable in exploring bifurcations and energy landscapes of the microstructured fluid models, and could lead to discovery of new stable/unstable configurations relevant to material design.\\

\textbf{Industrial Collaboration and Impact.} Throughout the project, I will engage with industry partners to guide applications of our research. In particular, the collaboration with ZOEEN S.R.L. and Quanta System S.p.A. will steer the development of the nemato-acoustic imaging concepts towards real biomedical devices. Likewise, insights from our kinetic theory and simulations may inform companies dealing with complex fluid formulations (for example, in liquid crystal display technology or novel metamaterials). By integrating fundamental mathematical advances with practical objectives, this research programme not only advances the theory of microstructured fluids but also paves the way for innovative technologies leveraging the unique properties of ordered fluids.
\begingroup
\renewcommand{\section}[2]{}%
\bibliographystyle{plain} % We choose the "plain" reference style
\bibliography{refs} % Entries are in the refs.bib file
\endgroup
\end{document}